% Szablon pracy inżynierskiej/magisterskiej
% Autor: Wojciech Domski <wojciech.domski@pwr.edu.pl>
%
% Oryginalna wersja rozwijana przez dra inż. Adama Ratajczaka
%
% Lista zmian:
% * 2022.06.28 Dodanie uwag i wskazówek.
% * 2022.03.14 Zmiana kodu jednostki
% * 2022.03.14 Wykorzystanie nowej klasy mgr.cls dla W12N
% * Uproszczenie szablonu oraz uzupełnienie wymaganych
% informacji

%Przykładowy plik ułatwiający złożenie projektu dyplomowego inżynierskiego.
%UWAGA: Generowany napis na stronie tytułowej o treści PROJEKT DYPLOMOWY INŻYNIERSKI został zaproponowany przeze mnie i nie jest, póki co, potwierdzony przez władze wydziału. Przed ostatecznym oddaniem tak złożonej pracy należy upewnić się jaka powinna być treść tego napisu. W momencie gdy uzyskam informację na temat treści tego napisu, dokonam niezbędnych zmian w źródłach.

\documentclass[eng,printmode]{mgr}
%opcje klasy dokumentu mgr.cls zostały opisane w dołączonej instrukcji

%poniżej deklaracje użycia pakietów, usunąć to co jest niepotrzebne
\usepackage{polski} %przydatne podczas składania dokumentów w j. polskim
%\usepackage[polish]{babel}%alternatywnie do pakietu polski, wybrać jeden z nich
\usepackage[utf8]{inputenc} %kodowanie znaków, zależne od systemu
\usepackage[T1]{fontenc} %poprawne składanie polskich czcionek

%pakiety do grafiki
\usepackage{graphicx}
\usepackage{subfigure}
\usepackage{psfrag}

%pakiety dodające dużo dodatkowych poleceń matematycznych
\usepackage{amsmath}
\usepackage{amsfonts}

%pakiety wspomagające i poprawiające składanie tabel
\usepackage{supertabular}
\usepackage{array}
\usepackage{tabularx}
\usepackage{hhline}

%różne pakiety
\usepackage{hyperref} % linki

%pakiet wypisujący na marginesie etykiety równań i rysunków zdefiniowanych przez \label{}, chcąc wygenerować finalną wersję dokumentu wystarczy usunąć poniższą linię
\usepackage{showlabels}

%definicje własnych poleceń
\newcommand{\R}{I\!\!R} %symbol liczb rzeczywistych, działa tylko w trybie matematycznym
\newtheorem{theorem}{Twierdzenie}[section] %nowe otoczenie do składania twierdzeń

%dane do złożenia strony tytułowej
\title{Budowanie map otoczenia dla robota mobilnego w środowisku ROS}
\engtitle{English title}
\author{Marcel Konieczny}
\supervisor{dr inż. Wojciech Domski, K29W12ND02}

%\date{2008} %standardowo u dołu strony tytułowej umieszczany jest bieżący rok, to polecenie pozwala wstawić dowolny rok

%poniżej jest lista kierunków i specjalności na wydziale elektroniki, należy wybrać właściwe lub dopisać jeśli nie ma odpowiednich
\field{Automatyka i Robotyka (AIR)}
\specialisation{Robotyka (ARR)}

%tutaj zaczyna się właściwa treść dokumentu
\begin{document}
    \bibliographystyle{plabbrv} %tylko gdy używamy BibTeXa, ustawia polski styl bibliografii

    \maketitle %polecenie generujące stronę tytułową
    \dedication{6cm}{To jest przykładowa treść opcjonalnej dedykacji, należy ją zmienić lub usunąć w całości polecenie \texttt{$\backslash$dedication}}

    \tableofcontents %spis treści

%poniżej znajduje się przykładowa treść dalszej części dokumentu, zainteresowanych zachęcam do rozszyfrowania frazy "Lorem ipsum" :)
    \chapter{Wstęp}

    Dokument ten zawiera wytyczne dotyczące pisania pracy dyplomowej.

    Dokument pracy dyplomowej powinien być rozwijany w systemie składania
    tekstu LaTeX.
    Zbiorczą paczkę instalacyjną na system Windows wraz z edytorem można
    znaleźć pod adresem
    \url{https://www.tug.org/protext/}
    zwiera ona wszystkie niezbędne narzędzia do wytwarzania dokumentu pracy
    dyplomowej
    w formacie pliku PDF.

    Jako szablon dokumentu należy wykorzystać klasę mgr.cls przygotowaną przez
    dr. inż. Adama Ratajczaka. Szablon dokumentu znajduje się w pliku
    \texttt{dyplom-inz-utf8-dos$\_$szablon.tex} (ten dokument).

    Praca powinna być rozwijana w webowym edytorze plików latexowych
    \url{https://www.overleaf.com}.
    Proszę o wysłanie zaproszenia do tego systemu w celu nanoszenia
    poprawek i komentarzy do Państwa prac dyplomowych.

    \section{Teza}
    \label{sec:Teza}

    Tutaj należy umieścić tezę pracy. Powinno być to jedno zdanie, które
    \textbf{bardzo} zwięźle ujmuje całą pracę. Do samej tezy należy
    odnieść się w samym podsumowaniu i pokazać, że w toku pracy została
    ona zrealizowana.


    \chapter{Konfiguracja środowiska symulacyjnego}

    \chapter{Przetwarzanie danych}
    \section{Przetwarzanie danych miernika laserowego}
    \section{Klasteryzacja danych}

    \chapter{Budowanie mapy otoczenia}


    \chapter{Podsumowanie}



\end{document}
\chapter{Elementy pracy}

Każdy rozdział powinien mieć w tym miejscu
element składający się z tekstu, któ\ref{eq:pitagoras}y np. będzie
wprowadzeniem do tego rozdziału. Zatem, w poniższym
rozdziale zostały opisane najczęściej występujące elementy
dokumentu i sposób ich właściwego wykorzystania.

\section{Ogólne uwagi}


\textbf{Praca powinna być pisana bezosobowo i w czasie przeszłym.}

Do własnej osoby należy odwoływać się jako "autor". Praca dyplomowa
jest pisana w czasie przeszłym ponieważ jest to efekt Państw pracy,
a nie plan, który należy dopiero wykonać.

Proszę unikać również trzeciej osoby liczby mnogiej, np.:
Wybraliśmy zestaw czujników .... -> Wybrano zestaw czujników.

Praca dyplomowa jak każda praca powinna zawierać wstęp,
rozwinięcie i zakończenie. Sam wstęp powinien być pisany według zasady ,,od ogółu do szczegółu''.
Najpierw należy wyjść od ogólnych sformułowań, a następnie stopniowo przejść
do problemu opisanego w pracy.

We wstępie należy określić czego będzie dotyczyć praca, obszerność tej
części
to co najmniej 4 strony. Należy również odnieść się do istniejących
rozwiązań,
tak aby usytuować Państwa pracę. W przypadku pracy dyplomowej, gdzie
znajduje się aspekt badawczy należy przedstawić tezę pracy -- jedno lub dwa
zdania, które w sposób syntetyczny przedstawia problem badawczy.
W zależności od problematyki, którą się
Państwo zajmujecie rozwinięcie może przybrać różny kształt.
Jednak powinno być ono przedstawione w sposób logiczny i spójny.

Przykładowo w pracy można znaleźć sformułowanie:

,,Jednym z problemów napotkanych w trakcie realizacji pracy była obsługa wyświetlacza.
Wyświetlacz o rozdzielczości 320x240 pikseli i 16-bitowej głębi kolorów wymaga bufora
ramki o rozmiarze 153.6kB. Cała pamięć RAM mikrokontrolera ma pojemność
zaledwie 128kB. Modyfikacje obrazu są więc zapisywane bezpośrednio do pamięci GRAM
wyświetlacza, co jednak spowalnia działanie programu i utrudnia wprowadzanie zmian.''

Zamienione w
,,Jednym z możliwych usprawnień jest poprawienie komfortu korzystania z wyświetlacza.
Celem uzyskania wyższej częstotliwości odświeżania wyświetlanego obrazu na
ekranie LCD należałoby wykorzystać inny mikrokontroler oferujący większą ilość
pamięci RAM. Taki zabieg pozwoliłby na buforowanie rysowanego obrazu, a
tym samym przyspieszenie jego wysyłania na ekran.''

Unikać nacechowania. Nie należy pisać pracy, gdzie
będą wykorzystywane kolokwializmy, albo nadmierne nacechowania.
Praca powinna przedstawiać fakty, a nie oddawać
osobiste poglądy autora.

W pracy należy przedstawić fakty, ale nie suche fakty.
Nie powinna być to dokumentacja techniczna. Fakty należy połączyć
ze sobą w ogólne wnioski.

Przykład:
Zastosowano dodatkowy ekran do kabla łączącego dwa moduły
ze sobą, bo czasem powstawały błędy w komunikacji.

Zdanie powinno wyglądać tak:
Zdecydowano się na zastosowanie dodatkowego ekranu
dla przewodu łączącego dwa moduły systemu, ponieważ
występowały błędy w komunikacji. Takie działanie
pozwoliło podnieść liczbę poprawnie przesyłanych
ramek danych za pośrednictwem łącza.

\begin{itemize}
    \item
    nie ma potocznych określeń: kabel,
    \item
    nie ma niepotrzebnych i nieprecyzyjnych
    określeń: dodatkowy, czasem,
    \item
    formalny język: nie ma ,,bo'',
    \item
    drugie zdanie prezentuje wymierny skutek
    zastosowanego działania.
\end{itemize}


\section{Rysunki i inne materiały}

Jeśli wykorzystujecie Państwo rysunki, które nie są Państwa własnością,
a pochodzą np. z książki należy w podpisie podać cytowanie.
To samo dotyczy wzorów i formuł matematycznych, konceptów, itd.
Na końcu podpisu nie wstawiamy kropki.

Zdjęcia/rysunki w pracy nie powinny mieć większego rozmiaru niż 150kB.
Jeżeli jest to możliwe rysunki powinny być w formacie wektorowym
(np. diagramy, schematy, itp.).

\section{Równania}

Odwołania do równań umieszczamy w nawiasach okrągłych.
Równanie traktujemy jako część zdania, czyli należy na
końcu postawić kropkę, bądź przecinek jeśli wymieniamy
nazwy zmiennych, które zostały użyte pierwszy raz. na przykład:

\begin{equation}
    \label{eq:pitagoras}
    a^2 + b^2 = c^2,
\end{equation}
gdzie $a$ to ... Należy zwrócić uwagę na przecinek po równaniu.
Równanie jest traktowane jako zdanie, które możemy kontynuować
w tekście pracy.

Aby odwołać się do równania należy od razu podać jego referencję.
W (\ref{eq:pitagoras}) przedstawiono przykładowa równanie.
Proszę unikać bezpośredniego wykorzystania słowa ,,równanie'', np.
w równaniu (\ref{eq:pitagoras}) widzimy wzór.

\section{Listy i wypunktowania}

Każde wypunktowania rozpoczyna się od małej litery, a kończy przecinkiem
oprócz ostatniego elementu, np. to jest wypunktowanie:
\begin{itemize}
    \item
    pozycja 1,
    \item
    pozycja 2,
    \item
    długa pozycja 3. Jeszcze dodatkowy opis,
    \item
    ostatnia pozycja.
\end{itemize}

W przypadku listy mamy:
\begin{enumerate}
    \item
    Pierwszy punkt.
    \item
    Drugi punkt.
    \item
    Ostatni punkt.
\end{enumerate}

\section{Myślnik}

W Latexu występują 3 rodzaje poziomych kresek: -, --, ---.
Pojedyncza kreska występuje przy automatycznym łamaniu wyrazów celem
jego przeniesienia do nowej linii.
Podwójna kreska pełni rolę myślnika.
Potrójna kreska zaznacza np. wypowiedź bohatera (dialogi w książkach
fabularnych).

\section{Odmiany obcojęzycznych nazwisk}

W przypadku, gdy odmienione nazwisko np. w języku angielskim
brzmi dobrze (jego wymowa nie jest łamana) to nie łączymy
nazwiska z końcówką poprzez apostrof.

Przykładowo:
Filtr Madgwicka, NIE Filtr Madgwick'a
Filtr Mahony'ego, NIE Filtr Mahonyego, NIE Filtr Mahoniego

\section{Odwołania}

W przypadku umieszczenia w pracy rysunków, tabel, fragmentów
kodu programów należy wykorzystywać odwołania do tych elementów.
Jeśli w treści pracy nie ma odwołania do np. rysunku, to
oznacza, że nie powinien się tam znaleźć i jest zbędny.

\section{Wdowy, bękarty, szewcy i sieroty}

W pracy należy unikać tego typu pozostałości, czy innych niedociągnięć.
Unikanie tego typu elementów pracy sprawi, że będzie ona wyglądałoa
dużo estetyczniej. W przypatku sierot problem można rozwiązać
wykorzystując tzw. twardą spację (tylda)
pomiędzy dwoma wyrazami i~cieszyć się efektem.

\chapter{Literatura}

Należy unikać bezpośredniego odwołania do bibliografii.

Książka \cite{SR01} opowiada o ...

Należy wykorzystać jednak odwołanie do pozycji w następującej
postaci.

Do wytwarzania oprogramowania dla mikrokontrolerów
można wykorzystać dedykowane oprogramowanie \cite{SR01}.

W pracach dyplomowych najczęściej pojawiają się odniesienia do książek,
artykułów naukowych, materiałów konferencyjnych oraz materiałów internetowych.
W przypadku tego ostatniego należy podać adres URL oraz datę ostatniego dostępu.
Proszę nie cytować Wikipedii, filmów na serwisach streamingowych (np. youtube).
Poniżej lista elementów, które powinny się znaleźć w przypadku cytowania
różnych rodzajów pozycji bibliograficznych:

Przykłady pozycji można znaleźć w pliku \textit{bibliografia.bib}.
Książka \cite{Kristic95}, artykuł w czasopiśmie \cite{Domski22},
artykuł konferencyjny \cite{Domski17},
materiał internetowy \cite{Sojourner},
inne np. instrukcja laboratoryjna \cite{SR01}.

\chapter{Podsumowanie}

W podsumowaniu należy odnieść się do wyników jakie udało się uzyskać
w trakcie pracy. Również tutaj jawnie należy się odnieść do tezy,
pokazać w sposób logiczny, że cel w niej przedstawiony został osiągnięty.

Zatem, podsumowanie powinno składać się z następujących części:
\begin{enumerate}
    \item
    Zdanie o wypełnieniu założeń pracy.
    \item
    Następnie kilka lub kilkanaście zdań o projekcie co zostało wykonane.
    \item
    Na końcu ewentualne dwa lub trzy problemy (maksymalnie), ta część powinna być
    dość krótka. Najlepiej podać propozycję rozwiązania.
    \item
    Koniec podsumowania powinien zawierać możliwe kierunki dalszego rozwoju.
    Jeżeli występowały w pracy problemy to proszę przedstawić owe problemy w
    trakcie pracy jako możliwe kierunki rozwoju, nie opisywać to jako problem
    i nie oznajmiać, że nie udało się czegoś wykonać.
\end{enumerate}

\newpage
\addcontentsline{toc}{chapter}{Załącznik A}
\appendix
\thispagestyle{empty}

\chapter*{Załącznik A}
\label{ch:ZalacznikA}

W pracy, podczas której wytworzyliście Państwo oprogramowanie, modele,
symulacje należy zawrzeć w pracy specjalny rozdział (dodatek), a
w nim zawrzeć informację o dołączeniu do pracy płyty CD z odpowiednią
adnotacją co do jej zawartości.

W pracy powinien pojawić się załącznik znajdujący się po bibliografii
jako rozdział nienumerowany.

Należy ogólnie opisać co znajduje się na dołączonej płycie CD.
Powinna to być lista (lub środowisko description).

Np.

Do pracy załączono płytę DVD zawierającą w~poszczególnych katalogach:

\begin{description}
    \item
    \texttt{/Praca$\_$inzynierska.pdf} –- wersja cyfrowa pracy,
    \item
    \texttt{/Kod$\_$zrodlowy} –- kod źródłowy oprogramowania do ...,
    \item
    \texttt{/PCB.zip} -- projekt płytki PCB,
    \item
    \texttt{/Czesci$\_$mechaniczne.zip} -- archiwum z częściami mechanicznymi wykorzystanymi
    w projekcie.

\end{description}


\addcontentsline{toc}{chapter}{Bibilografia} %utworzenie w spisie treści pozycji Bibliografia
\bibliography{bibliografia} % wstawia bibliografię korzystając z pliku bibliografia.bib - dotyczy BibTeXa, jeżeli nie korzystamy z BibTeXa należy użyć otoczenia thebibliography

%opcjonalnie może się tu pojawić spis rysunków i tabel
% \listoffigures
% \listoftables
\end{document}

